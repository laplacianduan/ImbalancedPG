\documentclass[11pt]{amsart}

\usepackage{graphicx,epsfig}
%\usepackage{multirow}
\usepackage{verbatim}
\usepackage{amsmath,amssymb,latexsym, amsfonts, amscd, amsthm,bbm,mathtools,natbib, bbm}
%\input{xy}
%\xyoption{all}
\usepackage[pdftex,bookmarks,colorlinks,breaklinks]{hyperref}
\hypersetup{linkcolor=blue,citecolor=blue,filecolor=cyan,urlcolor=blue}   % PDF hyperlinks, with coloured links
\usepackage{url}
%\usepackage{color}
%options for class are \documentclass[oneside, draft]{amsart}

\numberwithin{equation}{section}

\newcommand{\figref}[1]{\figurename~\ref{#1}}
\usepackage[usenames]{color}

\usepackage{parskip,bm}
\pagestyle {plain}
\usepackage{mhequ}

\makeindex
%
\setcounter{tocdepth}{2}
%\noindent



\voffset = -15pt \hoffset = -40pt \textwidth = 420pt \textheight
=600pt \headheight = 12pt \headsep = 0pt






\theoremstyle{plain}

\newtheorem{theorem}{Theorem}[section]
\newtheorem{conjecture}[theorem]{Conjecture}
\newtheorem{proposition}[theorem]{Proposition}
\newtheorem{corollary}[theorem]{Corollary}
\newtheorem{hypothesis}[theorem]{Hypothesis}
\newtheorem{assumption}[theorem]{Assumption}

\newtheorem{lemma}{Lemma}[section]
\newtheorem{question}[theorem]{Question}


\theoremstyle{definition}
\newtheorem{definition}[theorem]{Definition}
\newtheorem{remark}{Remark}[section]
\newtheorem{example}[theorem]{Example}
\newtheorem*{thm}{Theorem}
\newtheorem*{fact}{Fact}
%\newtheorem*{remark}{Remark}


%###############################
\def\eps{\varepsilon}
\def\m{\mathcal}
\def\mb{\mathbb}
\def\mr{\mathrm}

\def\dist{\rm{dist}}
\def\co{{\rm co }}
\def\bin{{\rm bin}}
\def\card{{\rm Card}}
\def\boot{\rm boot}
\def\act{\rm act}
\def\supp{{\rm supp}}
\def\tr{{\rm tr\,}}
\def\l{\left}
\def\r{\right}
\def\sinc{\mathop{\rm sinc}}
\def\vol{{\rm Vol}}
\DeclareMathOperator{\esssup}{ess\,sup}
\newcommand{\dotp}[2]{\left\langle #1, #2\right\rangle}
\newcommand{\argmin}{\mathop{\rm argmin~}}
\newcommand{\Var}{\mathop{\rm Var}}
\newcommand{\Cov}{\mathop{\rm Cov}}
\newcommand{\dom}{\mathop{\rm dom} \,}
\newcommand{\bs}[1]{\boldsymbol{#1}}
\newcommand{\ind}{\mathbbm{1}}

\newcommand{\bfY} {\mbox{\boldmath $Y$}}
\newcommand{\bfth} {\mbox{\boldmath $\theta$}}
\newcommand{\bfgam} {\mbox{\boldmath $\gamma$}}
\newcommand{\bfe} {\mbox{\boldmath $\epsilon$}}
\newcommand{\bfeb} {\mbox{\boldmath $\underline{\epsilon}$}}
\newcommand{\bfxi} {\mbox{\boldmath $\xi$}}


\newcommand{\abs}[1]{\left\vert#1\right\vert}
\newcommand{\set}[1]{\left\{#1\right\}}
\newcommand{\seq}[1]{\left<#1\right>}
\newcommand{\norm}[1]{\left\Vert#1\right\Vert}
\newcommand{\essnorm}[1]{\norm{#1}_{\ess}}

\newcommand \bbP{\mathbb{P}}
\newcommand \bbE{\mathbb{E}}
\newcommand \bbR{\mathbb{R}}
\newcommand \bbL{\mathbb{L}}
\newcommand{\mc}[1]{\mathcal{#1}}
\newcommand{\bb}[1]{\mathbb{#1}}
\newcommand{\be}{\begin{equs}}              
\newcommand{\ee}{\end{equs}}
\newcommand{\bone}[1]{\mathbbm{1}_{\{#1\}}}


\graphicspath{{./}{Graphics/}}


\def\T{{ \mathrm{\scriptscriptstyle T} }}
\def\Anirban{\textcolor{magenta}}


%###############################


\DeclareMathOperator{\sign}{sign}


\begin{document}

\title{Notes on threshold updates for ordinal probit}
\author{J. Johndrow}
\maketitle

The proposed algorithm for ordinal probit is described as:
\newline
\noindent \emph{CDA first integrates the joint distribution }
$$\pi(z,y,x,\beta,\gamma,y)= \prod_i  {\mathcal N} (z_i |x^T_i\beta,1)   \prod_{j=1} \frac{1 \{ \underset{i:y_i=j}{\max}(z_i) < \gamma_j< \underset{i:y_i=j+1}{\min} (z_i) \}}{\underset{i:y_i=j+1}{\min} (z_i) -\underset{i:y_i=j}{\max}(z_i) }$$ 
\emph{over $\gamma_2,\ldots \gamma_{k-1}$, where each uniform simply integrates to $1$.}
\newline

This was bothering me, since the MCMC algorithm is then identical to the binary probit algorithm. I think I have now found the problem. You need
\be
p(z_i \mid x_i, \beta, y_i) = \int_{\gamma_1,\ldots,\gamma_{K-1}} &p(z_i \mid \beta, y_i, \gamma) p(\gamma \mid X,y,\beta) \\
= \int_{\gamma_1,\ldots,\gamma_{K-1}} &\frac{\exp(-(z_i-x_i\beta)^2/2)}{\sqrt{2\pi} \{\Phi(\gamma_{y_i}-x_i\beta)-\Phi(\gamma_{y_i-1}-x_i\beta)\}} \\
&\times \bone{z_i \in [\gamma_{y_i-1},\gamma_{y_i}]}  p(\gamma \mid y, X, \beta) d\gamma,
\ee
where $p(\gamma \mid X, y, \beta) = p(\gamma_1,\ldots,\gamma_{K-1} \mid X,y,\beta)$ is the distribution of $\gamma$ given data and $\beta$, which we don't know. Regardless of what that is, this is not a normal distribution with mean $x_i \beta$ and variance 1, since you have to integrate the normalizing constant 
\be
\{ \Phi(\gamma_{y_i}-x_i\beta)-\Phi(\gamma_{y_i-1}-x_i\beta) \}^{-1}
\ee
over $\gamma$ to obtain $p(z_i \mid x_i, \beta, y_i)$, which is what you're trying to use in deriving your algorithm.



\end{document}
